\iffalse
Author: Kacper Gawroński

This file is part of Project Framework.

Project Framework is free software: you can redistribute it and/or
modify it under the terms of the GNU General Public License as published
by the Free Software Foundation, either version 3 of the License, or
(at your option) any later version.

Project Framework is distributed in the hope that it will be
useful, but WITHOUT ANY WARRANTY; without even the implied warranty of
MERCHANTABILITY or FITNESS FOR A PARTICULAR PURPOSE.  See the
GNU General Public License for more details.

You should have received a copy of the GNU General Public License
along with Project Framework. If not, see
https://www.gnu.org/licenses/
\fi

\documentclass[a4paper]{article}
\usepackage[utf8]{inputenc}
\title{Documentation\newline for Project Framework}
\author{Kacper Gawroński}
\begin{document}
\maketitle
\section{Introduction}
Source files can be found on https://github.com/KacperGawronski/project\_framework .\newline
Project started as example usage of socket.h.\\
Next step was implementing handling of http request using Lua scripting
language. When it was done, I noticed that it can script page generation
structure and handle all requests. Project is on GNU GPL license, 
however, some external libraries might be on other licenses.
\section{Dependencies}
Basic project require:\\
\begin{itemize}
\item{POSIX compiliant system} Basically project is targeted on GNU/Linux
OS - Debian.
\item{Lua5.3 libraries and headers}
\end{itemize}
Example worker also require:
\begin{itemize}
\item{MariaDB dev files (C connector)}
\item{jansson library and header files} for database api
\end{itemize}
To run example project it is also required to have MariaDB installed,
with example database from:\newline
https://github.com/datacharmer/test\_db\newline
installation process can be found on:\newline
https://www.ibm.com/developerworks/library/l-lpic1-105-3/index.html\newline
then add user exampleuser:\newline
CRATE USER 'exampleuser'@'localhost' IDENTIFIED BY '123';\newline
GRANT SELECT ON employees.* TO 'exampleuser'@'localhost';
\newpage
\section{Structure}
\subsection{webserver.c}
webserver.c contains code that allow handling connections.
It makes listening socket (by default on port 9090), and forwards
accepted connections to threads based on worker function. It passes
struct worker\_arg containing all
required data to process connection.\\
By default, threre is limitation 100 for number of threads, it's
controlled by semaphore.
\subsection{worker.c}
function worker() handles connections. It reads sent data (by default
only 10240 bytes), initializes Lua library as interpreter, and pushes
on it additional C functions.\\
Example of adding C function to Lua scripting level:\newline
\begin{verbatim}
		lua_pushcfunction(L,generate_menu);
		lua_setglobal(L,"generate_menu");
\end{verbatim}
Function should be defined as:\\
\begin{verbatim}
	int name(lua_State * L){ (...) return number_of_returned_values }
\end{verbatim}
Example functions are placed in menu.c and mariadb\_connector.c files.
\newpage
\subsection{Page structure}
\paragraph{Example page} is described in following files:
\begin{itemize}
	\item{example.lua}\newline\indent Placed in app/pages directory.
File contains page description - header definitions, javascript files to
use and body
	\item{example.js}\newline\indent Placed in app/javascript directory.
File contains javascript which is run on page load.
	\item{style.css}\newline\indent Global css file placed in app/css
directory.
	\item{api.lua}\newline\indent File describing json api, as for
api.json?[smth] requests.
	\item{page\_template.lua} File placed in app directory. It is used
for processing every lua file in app/pages directory. 
It contains page structure definition.
\end{itemize}
\paragraph{Generally}
	\begin{itemize}
		\item{worker} directory contains basic files for processing and 
handling requests - especially forwarding them to Lua interpreter.
		\item{app} directory contains definitions of what should be done
with request.
		\item{app/GET.lua} Main file for processing GET request. It
describes actions taken on specific GET requests. In this project, for
example, it does file app/pages/filename.lua for request like
"/page?filename"
		\item{app/page\_template.lua} returns function which should be done
by every page using definied table as argument.
		\item{app/pages} directory contains files describing pages, in
format definied by page\_template.lua code.
		\item{app/pages/pagename.lua} is file describing pagename site.
It should define table as used by page\_template.lua. Link to it will be
generated by generate\_menu() function as page?pagename . 
To add link in html you should simple write page?filename without
extension (it will be added in code, prevents a little from hacking).
		\item{app/javascript} is main directory for javascript .js files.
It won't automatically include js for every page, it needs to be definied
in page file.
		\item{app/css} directory contains css files, and description of
files included in every page in requires.txt.
		\item{json\_api} contains file api.lua, which is called on GET
/api.json?[smth] request. It is example of additional aplication.
	\end{itemize}

\end{document}
